\documentclass[t, aspectratio=169]{beamer}
\usepackage[utf8]{inputenc}
\usepackage[T1]{fontenc}
\usepackage{xcolor}
\usepackage{hyperref}
\usepackage{tikz}
\usepackage{pgfplots}
\usepackage{adjustbox}
\usepackage{tabularx}
% 
\usepackage{enumitem}
\usepackage{fontspec}
\usefonttheme{serif} % default family is serif

\title{Bayesian data analysis with TensorFlow Probability}
\date{DataScienceConference Europe 2020}
\author{Simeon Carstens \& Dorran Howell, Tweag I/O}
\newcommand{\todo}{\textcolor{red}{\textbf{TODO}}}
\renewcommand{\d}{\mathrm{d}}
\setmainfont{Roboto}
\usetheme{tweag}

\begin{document}

\begin{frame}
  \titlepage
\end{frame}


\begin{frame}
  \frametitle{Your hosts}
  \begin{tcolorbox}[title=Simeon will give the presentation]
    \begin{minipage}{0.3\textwidth}{
        \includegraphics[width=0.15\paperwidth]{images/simeon}}
    \end{minipage}
    \begin{minipage}{0.6\textwidth}
    \begin{itemize}
    \item background in computational biology
    \item Data Scientist at Tweag since 2019
    \end{itemize}
    \end{minipage}
  \end{tcolorbox}

  \begin{tcolorbox}[title=Dorran will happily answer questions]
    \begin{minipage}{0.3\textwidth}{
        \includegraphics[width=0.15\paperwidth]{images/dorran}}
    \end{minipage}
    \begin{minipage}{0.6\textwidth}
    \begin{itemize}
    \item previous positions in geophysics
    \item Data Scientist at Tweag since 2019
    \end{itemize}
    \end{minipage}
  \end{tcolorbox}
\end{frame}


\begin{frame}
  \frametitle{Tweag I/O}
  \todo \\
  Tweag I/O is a software innovation lab and consultancy based in Paris with employees all around the world.\\
  We specialize in
  \begin{itemize}
  \item software engineering, with a focus on functional programming
  \item DevOps, with a focus on reproducible software systems and builds
  \item data science
  \end{itemize}
\end{frame}


\begin{frame}
  \frametitle{What you're in for}
  This tutorial consists of alternating blocks of
  \begin{itemize}
  \item theory / example slides
  \item practical examples on either external websites or Google Colab notebooks. Links are provided at {\centering \url{https://github.com/tweag/tutorial-dsc-2020/}}
  \end{itemize}

  Requirements:
  \begin{itemize}
  \item a Google account (for the practical exercises)
  \item elementary knowledge in probability theory and statistics
  \end{itemize}
\end{frame}


\begin{frame}
  \frametitle{Bayesian vs frequentist probabilities}
  Example: fair coin flip with $p(\mathrm{head}) = p(\mathrm{tail}) = \frac{1}{2}$
  \begin{block}{Frequentist probability}
    \begin{description}
    \item[$p(\mathrm{"flip\ results\ in\ head"}|b=\frac{1}{2})=\frac{1}{2}$:] \hfill \\ $\frac{\mathrm{\# \ of \ heads}}{\mathrm{\# \ total \ flips}}$ for $\infty$ many fair coin flips
    \end{description}
  \end{block}
  \begin{block}{Bayesian probability}
    \begin{description}
    \item[$p(\mathrm{"flip\ results\ in\ head"}|b=\frac{1}{2})=\frac{1}{2}$:] \hfill \\ measure of \textit{belief} in the statement ``flip results in head'' given single fair coin flip
    \end{description}
  \end{block}
  \todo: this slide looks complicated and is aesthetically offputting
\end{frame}


\begin{frame}
  \frametitle{Prior beliefs}
  Assume: unknown bias $b$
  \begin{block}{Prior probability}
    Encodes prior belief in $b$ \textit{before} flipping the coin
  \end{block}
  What is known about $b$?
  \begin{itemize}
  \item $b$ is a probability: $0 \leq b \leq 1$
  \item most coins are fair
  \end{itemize}
  \begin{itemize}
  \item[$\rightarrow$] choose prior distribution defined between $0$ and $1$, with maximum at and symmetric around $b=\frac{1}{2}$
  \end{itemize}
  Example:
  \begin{equation*}
    b \sim \mathrm{Beta}(\alpha=2,\beta=2)
  \end{equation*}
  with
  \begin{equation*}
    \mathrm{Beta}(x;\alpha, \beta) \propto x^{\alpha-1}(1-x)^{\beta-1}
  \end{equation*}
\end{frame}


\begin{frame}
  \frametitle{Posterior belief}
  Now: flip coin one time, result: $\mathrm{head}$
  \begin{block}{Posterior belief}
    \begin{description}
      \item[$p(b|\mathrm{head})$:] updated prior belief after obtaining new data
    \end{description}
  \end{block}
  \todo: some illustration of shifting distribution
\end{frame}


\begin{frame}
  \frametitle{Update rule}
  \begin{block}{Bayes' theorem}
    \begin{equation*}
      p(A|B) = \frac{p(B|A) \times p(A)}{p(B)}
    \end{equation*}
    (easily derived from rules for conditional probabilities)
  \end{block}
  In data analysis:
  \begin{equation*}
    \underbrace{p(x|D,I)}_{\mathrm{posterior}} = \underbrace{p(D|x,I)}_{\mathrm{likelihood}} \times \underbrace{p(x|I)}_{\mathrm{prior}} / \underbrace{p(D|I)}_{\mathrm{evidence}}
  \end{equation*}
  \begin{description}
  \item[$x$:] model parameter
  \item[$D$:] data
  \item[$I$:] prior information (often not made explicit)
  \end{description}
\end{frame}


\begin{frame}
  \frametitle{Likelihood}
  $p(D|x)$: probability of the data given fixed model parameters\\
  $\rightarrow$ models data-generating process\\
  \bigbreak
  In our case:
  \begin{equation*}
    p(k|b) = b^k(1-b)^{k-1} = \mathrm{Ber}(k;b)
  \end{equation*}
  with
  \begin{equation*}
    k = \begin{cases} 0:& \mathrm{tail}\\
      1:& \mathrm{head}
    \end{cases}
  \end{equation*}
  % \begin{block}{Likelihood function}
  %   Consider data fixed, but variable parameters:
  %   \begin{equation*}
  %     l(x; D)=L(D|x)
  %   \end{equation*}
  % \end{block}
  \todo: graph of likelihood function
\end{frame}


\begin{frame}
  \frametitle{Evidence}
  \onslide<1->
  \begin{description}
  \item[$p(D) = \int \d x\ p(D|x)p(x)$:]\hfill \\
    normalization constant (long story...)
  \end{description}
  In our case:
   \begin{tikzpicture}
     \node (content2) at (0,0) {\begin{minipage}{\paperwidth}
  \begin{align*}
    p(k=1) &= \int_0^1 \d b\ L(k=1|b) p(b) \\
           &= \int_0^1 \d b\ \left.\mathrm{Ber}(k; b) \times \mathrm{Beta}(k;\alpha=2, \beta=2)\right\vert_{k=1} \\
           &= \int_0^1 \d b\ \left.b^{k}(1-b)^{k-1} b (1-b)\right\rvert_{k=1} \\
           &\;\;\vdots \\
           &=\frac{1}{12}
  \end{align*}\end{minipage}};
     \onslide<2->\node[align=center,red,font={\fontsize{50}{50}\bfseries}, rotate=45] at (content2.center) {YIKES};
   \end{tikzpicture}
\end{frame}


\begin{frame}
  \frametitle{Update rule}
  In our coin flip example:\\
  \begin{minipage}{0.1\paperwidth}
    prior
  \end{minipage}
  \hspace{0.05\paperwidth}
  \begin{minipage}{0.3\paperwidth}
    \begin{align*}
      p(b) &= \mathrm{Beta}(b; \alpha=2, \beta=2) \\
           &\propto b^{\frac{1}{2}} (1-b)^{\frac{1}{2}}
    \end{align*}
  \end{minipage}
  \begin{minipage}[t][][b]{0.3\paperwidth}
    \begin{adjustbox}{width=0.7\textwidth}
      \begin{tikzpicture}
        \begin{axis}[
          axis lines = left,
          xlabel = $b$,
          ylabel = {$p(b)$},
          width = 0.35\paperwidth
          ]
          \addplot [
          domain=0:1, 
          samples=100, 
          color=black,
          ]
          {x^0.5 * (1 - x)^0.5};
        \end{axis}
      \end{tikzpicture}
    \end{adjustbox}
  \end{minipage}

  \begin{minipage}{0.1\paperwidth}
    likelihood
  \end{minipage}
  \begin{minipage}{0.3\paperwidth}
    \begin{align*}
      p(D|b) &= \mathrm{Bern}(k=1; b) \\
             &= b
    \end{align*}
  \end{minipage}
  \begin{minipage}{0.3\paperwidth}
  \begin{adjustbox}{width=0.7\textwidth}
    \begin{tikzpicture}
      \begin{axis}[
        axis lines = left,
        xlabel = $b$,
        ylabel = {$L(D|b)$},
        width = 0.35\paperwidth
        ]
        \addplot [
        domain=0:1, 
        samples=100, 
        color=black,
        ]
        {x};
      \end{axis}
    \end{tikzpicture}
    \end{adjustbox}
  \end{minipage}
  
  \begin{minipage}{0.1\paperwidth}
    posterior
  \end{minipage}
  \begin{minipage}{0.3\paperwidth}
    \begin{align*}
      p(b|D) &\propto p(D|b)\times p(b) / p(D)\\
             &=\mathrm{Beta}(b; \alpha=3, \beta=2) \\
             &\propto b^2(1-b)
    \end{align*}
  \end{minipage}
  \begin{minipage}[t]{0.3\paperwidth}
  \begin{adjustbox}{width=0.7\textwidth}
    \begin{tikzpicture}
      \begin{axis}[
        axis lines = left,
        xlabel = $b$,
        ylabel = {$p(b|D)$},
        width = 0.35\paperwidth
        ]
        \addplot [
        domain=0:1, 
        samples=100, 
        color=black,
        ]
        {x^2 * (1 - x) * 12};
      \end{axis}
    \end{tikzpicture}
    \end{adjustbox}
  \end{minipage}
\end{frame}

\end{document}
